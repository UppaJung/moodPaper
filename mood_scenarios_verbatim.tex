




Researchers at Facebook want to study whether users are more likely to share positive (happy) thoughts if their friends have been posting positive thoughts, and whether they are more likely to share negative (unhappy) thoughts if their friends have been sharing negative thoughts.
\begin{itemize}
\item To increase the proportion of positive posts in some users' news feeds, the researchers will randomly exclude some fraction of friends' negative posts each time the news feed is loaded.
\item To increase the proportion of negative posts in some users' news feeds, the researchers will randomly exclude some fraction of friends' positive posts each time the news feed is loaded.
\item The researchers will use an automated algorithm to measure whether users' posts are of a positive or negative mood.
\item The researchers will publish the anonymized aggregate results of the experiment in a scientific paper.
\item Participants will not be identified and will remain anonymous.
\end{itemize}
If the researchers are not allowed to perform this experiment, they will not be able to make a valid scientific determination of whether users' moods are affected by the moods of their friends' posts. Therefore, the researchers will not be able to produce features that might protect the moods of psychologically-vulnerable users.




Researchers at Twitter want to study whether users are more likely to share positive (happy) thoughts if their friends have been posting positive thoughts, and whether they are more likely to share negative (unhappy) thoughts if their friends have been sharing negative thoughts.
\begin{itemize}
\item To increase the proportion of positive posts in some users' news feeds, the researchers will randomly exclude some fraction of friends' negative posts each time the news feed is loaded.
\item To increase the proportion of negative posts in some users' news feeds, the researchers will randomly exclude some fraction of friends' positive posts each time the news feed is loaded.
\item The researchers will use an automated algorithm to measure whether users' posts are of a positive or negative mood.
\item The researchers will publish the anonymized aggregate results of the experiment in a scientific paper.
\item Participants will not be identified and will remain anonymous.
\end{itemize}
If the researchers are not allowed to perform this experiment, they will not be able to make a valid scientific determination of whether users' moods are affected by the moods of their friends' posts. Therefore, the researchers will not be able to produce features that might protect the moods of psychologically-vulnerable users.




Researchers at a social network want to study whether users are more likely to share positive (happy) thoughts if their friends have been posting positive thoughts, and whether they are more likely to share negative (unhappy) thoughts if their friends have been sharing negative thoughts.
\begin{itemize}
\item To increase the proportion of positive posts in some users' news feeds, the researchers will randomly exclude some fraction of friends' negative posts each time the news feed is loaded.
\item To increase the proportion of negative posts in some users' news feeds, the researchers will randomly exclude some fraction of friends' positive posts each time the news feed is loaded.
\item The researchers will use an automated algorithm to measure whether users' posts are of a positive or negative mood.
\item The researchers will publish the anonymized aggregate results of the experiment in a scientific paper.
\item Participants will not be identified and will remain anonymous.
\end{itemize}
If the researchers are not allowed to perform this experiment, they will not be able to make a valid scientific determination of whether users' moods are affected by the moods of their friends' posts. Therefore, the researchers will not be able to produce features that might protect the moods of psychologically-vulnerable users.




Researchers at Facebook want to study whether users are more likely to share positive (happy) thoughts if their friends have been posting positive thoughts, and whether they are more likely to share negative (unhappy) thoughts if their friends have been sharing negative thoughts.
\begin{itemize}
\item To increase the proportion of positive posts in some users' news feeds, the researchers will randomly exclude some fraction of friends' negative posts each time the news feed is loaded.
\item To increase the proportion of negative posts in some users' news feeds, the researchers will randomly exclude some fraction of friends' positive posts each time the news feed is loaded.
\item The researchers will use an automated algorithm to measure whether users' posts are of a positive or negative mood.
 \item Participants will not be identified and will remain anonymous.
\end{itemize}
If the researchers are not allowed to perform this experiment, they will not be able to make a valid scientific determination of whether users' moods are affected by the moods of their friends' posts. Therefore, the researchers will not be able to produce features that might protect the moods of psychologically-vulnerable users.




Researchers at Facebook want to study whether users are more likely to share positive (happy) thoughts if their friends have been posting positive thoughts, and whether they are more likely to share negative (unhappy) thoughts if their friends have been sharing negative thoughts.
\item To increase the proportion of positive posts in some users' news feeds, the researchers will randomly exclude some fraction of friends' negative posts each time the news feed is loaded.
\item To increase the proportion of negative posts in some users' news feeds, the researchers will randomly exclude some fraction of friends' positive posts each time the news feed is loaded.
\item The researchers will use an automated algorithm to measure whether users' posts are of a positive or negative mood.
\item The researchers will publish the anonymized aggregate results of the experiment in a scientific paper.
\item Participants will not be identified and will remain anonymous.
\end{itemize}
If the researchers are not allowed to perform this experiment, they will not be able to make a valid scientific determination of whether users' moods are affected by the moods of their friends' posts.




Researchers at Facebook want to study whether users are more likely to share negative (unhappy) thoughts if their friends have been sharing negative thoughts.
\begin{itemize}
 \item To increase the proportion of negative posts in some users' news feeds, the researchers will randomly exclude some fraction of friends' positive posts each time the news feed is loaded.
\item The researchers will use an automated algorithm to measure whether users' posts are of a positive or negative mood.
\item The researchers will publish the anonymized aggregate results of the experiment in a scientific paper.
\item Participants will not be identified and will remain anonymous.
\end{itemize}
If the researchers are not allowed to perform this experiment, they will not be able to make a valid scientific determination of whether users' moods are affected by the moods of their friends' posts. Therefore, the researchers will not be able to produce features that might protect the moods of psychologically-vulnerable users.




Researchers at Facebook want to study whether users are more likely to share positive (happy) thoughts if their friends have been posting positive thoughts .
\begin{itemize}
\item To increase the proportion of positive posts in some users' news feeds, the researchers will randomly exclude some fraction of friends' negative posts each time the news feed is loaded.
 \item The researchers will use an automated algorithm to measure whether users' posts are of a positive or negative mood.
\item The researchers will publish the anonymized aggregate results of the experiment in a scientific paper.
\item Participants will not be identified and will remain anonymous.
\end{itemize}
If the researchers are not allowed to perform this experiment, they will not be able to make a valid scientific determination of whether users' moods are affected by the moods of their friends' posts. Therefore, the researchers will not be able to produce features that might protect the moods of psychologically-vulnerable users.




Researchers at Facebook want to study whether users are more likely to share positive (happy) thoughts if their friends have been posting positive thoughts, and whether they are more likely to share negative (unhappy) thoughts if their friends have been sharing negative thoughts.
\begin{itemize}
\item To increase the proportion of positive posts in some users' news feeds, the researchers will randomly include additional positive posts that would otherwise have been deemed insufficiently relevant or unimportant.
\item To increase the proportion of negative posts in some users' news feeds, the researchers will randomly include additional negative posts that would otherwise have been deemed insufficiently relevant or unimportant.
\item The researchers will use an automated algorithm to measure whether users' posts are of a positive or negative mood.
\item The researchers will publish the anonymized aggregate results of the experiment in a scientific paper.
\item Participants will not be identified and will remain anonymous.
\end{itemize}
If the researchers are not allowed to perform this experiment, they will not be able to make a valid scientific determination of whether users' moods are affected by the moods of their friends' posts. Therefore, the researchers will not be able to produce features that might protect the moods of psychologically-vulnerable users.




Researchers at Facebook want to study whether users are more likely to share positive (happy) thoughts if their friends have been posting positive thoughts .
\begin{itemize}
\item To increase the proportion of positive posts in some users' news feeds, the researchers will randomly include additional positive posts that would otherwise have been deemed insufficiently relevant or unimportant.
 \item The researchers will use an automated algorithm to measure whether users' posts are of a positive or negative mood.
\item The researchers will publish the anonymized aggregate results of the experiment in a scientific paper.
\item Participants will not be identified and will remain anonymous.
\end{itemize}
If the researchers are not allowed to perform this experiment, they will not be able to make a valid scientific determination of whether users' moods are affected by the moods of their friends' posts. Therefore, the researchers will not be able to produce features that might protect the moods of psychologically-vulnerable users.




Researchers at Facebook want to study whether users are more likely to share positive (happy) thoughts if their friends have been posting positive thoughts, and whether they are more likely to share negative (unhappy) thoughts if their friends have been sharing negative thoughts.
\begin{itemize}
\item To increase the proportion of positive posts in some users' news feeds, the researchers will randomly exclude some fraction of friends' negative posts each time the news feed is loaded.
\item To increase the proportion of negative posts in some users' news feeds, the researchers will randomly exclude some fraction of friends' positive posts each time the news feed is loaded.
\item The researchers will use an automated algorithm to measure whether users' posts are of a positive or negative mood.
\item The researchers will publish the anonymized aggregate results of the experiment in a scientific paper.
\item Participants will not be identified and will remain anonymous.
\item The researchers promise in writing that the research findings will be used only to further science and improve the product for users. The results will not be used to improve Facebook's advertising algorithms.
\end{itemize}
If the researchers are not allowed to perform this experiment, they will not be able to make a valid scientific determination of whether users' moods are affected by the moods of their friends' posts. Therefore, the researchers will not be able to produce features that might protect the moods of psychologically-vulnerable users.
