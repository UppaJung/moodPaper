%\section{Discussion}
%Some critics of Facebook's emotional contagion experiment have argued that it should have received the same level of scrutiny that would be required of an experiment run at a university~\cite{WSJ:Criticism,CBC:Criticism,Ithaca:Criticism,WashingtonPost:Criticism}.
%Without stepping into the debate of whether or when an institutional review board is \emph{necessary}, we believe our results may provide insight to the question of whether such review would be \emph{sufficient}.  Respondents who reported being unaware of Facebook's experiment were more concerned for participants of two university-run studies which had been approved by ethics boards.  They were also at least as reluctant to see those two university experiments proceed as they were about Facebook's experiment.

\section{Limitations}
Our survey had a number of limitations that are important to consider when examining our results.

Our experiment was designed to gauge differences in how respondents felt about experimental manipulations, and excluded the question of consent by eliding discussion of it.  We did this because we believed respondents would be unlikely to disapprove of studies that participants knowingly consented to.  Our decision to elide the lack of consent from all scenarios facilitates \emph{relative} comparisons between experimental scenarios (such as sending phishing messages, removing items from news feeds, spoofing operating systems dialogs) and among variants of the Facebook scenario.  However, the consequence of this elision is that absolute levels of disapproval and concern that we report may underestimate the actual levels of concern that would be present had we explicitly stated that researchers did not obtain participant consent.  (Had our scenarios mentioned that researchers did not obtain consent, we would have also raised the expectation of consent among those who might otherwise not expected it.)

The process of compressing an experimental scenario into a short description introduces a number of other possible sources of error.  In crafting these descriptions, we may have failed to anticipate which facts would be influential in respondents ethical decision making.  We may have incorrectly interpreted information about an experimental design.  As two authors were researchers on two of the studies described, we may have been subject to subconscious biases (or, a skeptical reader may reasonably suspect, conscious ones).

In order to reach a large number of respondents in a very short time, our survey relied on a convenience sample: workers on Amazon's Mechanical Turk crowdsourcing service.  These individuals tend to be more tech savvy than the rest of the population.  They also likely find themselves participating in far more research experiments, and interacting with researchers, than members of the general population.\sscomment{Aleecia wants us to dig through data here.  She's right!}  Some may be reliant on research studies for income and more forgiving of transgressions so long as they are paid.  While these workers are an excellent group to reach out to in order to gauge the response research studies in which participants will be workers on Amazon's Mechanical Turk (e.g., studies~\TreatmentNum{OSCS} and~\TreatmentNum{WD}), the demographic differences are more problematic for examining research in which participants will be drawn from other populations.

Even if survey respondents closely resemble those who would be participants in research scenarios, there's no way to be certain that their responses to hypothetical questions about an experimental design will match how they would feel if they were to actually participate.  While the respondents of the survey had reasonable resemblance to the self-reported responses of experimental participants for Scenario~\TreatmentNum{WD}, this is no way guarantees surveys will be predictive for others studies.

Finally, respondents were not required to have any prior background in ethics, ethics training, or knowledge of laws and regulations that govern research ethics (e.g., the common rule); nor did we provide them with any such background or training.  This was by design.  Ethical controversies can occur when there is a disconnect between what cutting-edge research can be approved within the existing regulatory regime and what the public considers acceptable.  Further, the rules give ethics boards considerable discretion to determine whether waiving rules such as participant consent is in the public interest.

\section{Discussion}
Much of the debate over Facebook's emotional contagion experiment has focused on rules and process. Following the controversy, the lead author of Facebook's experiment promised that the company's ethical-compliance process for research had evolved and improved~\cite{Kramer2014:ResponseToControversy}. Some industrial research labs, including Microsoft Research (which employs one of the authors of this paper) already have established experimental review boards (on which the author serves).

Regardless of what processes evolve to govern the set of individuals who must decide whether research is approved or rejected, those tasked with making the decisions will have tough choices.  Most of the rules that govern research, such as the requirement for participant consent, give review boards considerable discretion.  Ethics boards often have very little data with which to understand the implications of exercising their discretion.

Thus, despite the limitations of ethical-response surveys, we would have to imagine that those making decisions -- whether for industrial research or research at universities -- would prefer to have data with known limitations to no data at all.  The costs of running such studies can be amortized over a number of experimental designs.  It is our hope that ethical-response surveys become a standard tool for use when researchers propose new types of experiments about which the the reaction of participants and the public is unknown.
%put yourself in the mind of whoever will be regulating facebook's experiments in the future, and you decide whether you would want this kind of data before launching your next study
