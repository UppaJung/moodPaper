\begin{figure}[t]
\fcolorbox{boxbordercolor}{boxbgcolor}{
\begin{minipage}{0.95\columnwidth}
\footnotesize
\begin{packed_quotation}
\footnotesize
In this survey, you will be ask you about five hypothetical scientific experiments.  For each, we will ask you to:

\begin{packed_itemize}
\item \textit{Carefully} read an abstract description of the experiment (350 words or fewer).
\item Answer 4 multiple-choice questions about each experiment.
\item Optionally provide short explanations of your answers.
\end{packed_itemize}
Finally, we will ask you some brief demographic questions at the end of the study.  All personal information (e.g., age) is optional.   Your responses will be kept anonymous, though we reserve the right to copy or quote the responses you provide.

The entire survey should take \textbf{under 10 minutes of your time} and pays \textbf{\$1.00}.

This survey is part of a research project being conducted by \anonymized{the The Ethical Research Project}.

If you have any questions, please feel free to contact us at: team@ethicalresearch.org
\end{packed_quotation}
\end{minipage}}
\caption{We provided the following description to prospective respondents.  The error ``you will be ask you'' in place of ``we will ask you'' is in the original and not a transcription error.  Fortunately, we described the task again correctly in the first page of the survey.}
\label{fig:offer}
\end{figure}
\section{Experimental procedure}
We used a single Human Intelligence Task on Amazon's Mechanical Turk to prevent the same worker account from taking the survey twice (though we cannot guarantee some workers with multiple accounts did not do so).  We restricted workers to those coming from the United States.

After brief instructions, we presented five experimental scenarios in random order (randomized for each participant).  Four of the scenarios were the same for each respondent, but we randomly assigned each respondent one of ten variants of the Facebook experiment.  We then asked follow-up questions.

\subsection{Recruiting and instructions}
We offered a Human Intelligence Task (HIT) on Mechanical Turk in which we presented prospective respondents with the offer in Figure~\ref{fig:offer}.

We presented participants who accepted the HIT with the following instructions:

\begin{packed_quotation}\footnotesize
Each of the following five pages will contain a description of a hypothetical scientific experiment, followed by questions about that experiment.
In order to answer the questions, please read the description of each experiment carefully.
\end{packed_quotation}
\subsection{Questions for each scenario}
We randomized the order of the experimental scenarios, but kept the ordering of questions and response options consistent.

The first question that followed the description of each scenario was one we designed to measure respondents' \emph{concern} for those participating in the experiment.  We asked: ``If someone you cared about were a candidate participant for this experiment, would you want that person to be included as a participant?''

We asked respondents about someone they care about, as opposed to themselves, because they might be more comfortable imagining others to be vulnerable and needing protection, whereas they might not want to admit being vulnerable themselves.  We provided the option to respond ``Yes'', ``I have no preference'', or ``No''.  We designed these options to be ordinal: from least concerned to most concerned.  We asked this concern question first in hopes that it would give respondents a chance to humanize potential participants and think about the consequences of the experiment on them.

We designed the second question to gauge whether respondents would disapprove of the experiment.  We asked, ``Do you believe the researchers should be allowed to proceed with this experiment?''  We offered four options, again ordered from most approving to least approving with the first option being ``Yes'' (on the left) and the last ``No'' (the fourth option, on the right).  We included the second option, ``Yes, but with caution'', for respondents who did not want to disapprove of a experiment but feared that an unambiguous ``yes'' would relieve researchers to their duty to take their ethical duties seriously.  The option in the third position from the left, between the two ``Yes'' options and ``No'', was ``I'm not sure.''  We treat this an ordinal value between the yes and the no options as the respondent is unable to commit to either and is therefore likely to be somewhere in between.

For each of the first two questions, we gave respondents a free-response field in which to explain their answers.

We also asked respondents ``Are you aware of having ever participated in such a study?'' and ``Are you aware of a study like this one having been performed by researchers in the past? (For example, have you have heard about it in the news or learned about it in a class?)''.  The answers options, from left to right, were ``Yes'' and ``No''.

\subsection{Closing questions}
After collecting respondents' responses to the five experimental scenarios, we asked the following questions about respondents' demographics and about factors that might influence their opinions:
%
\begin{packed_itemize}
\footnotesize
\item What year were you born?\\ (please use a four-digit year, or 'd' if you decline to answer)
\item
What is your gender?
\\  \{Male;Female;I'm uncomfortable answering\}
\item
What is your occupation?
\item
Have you ever purchased goods advertised via an unsolicited marketing email?
\\ \{Yes;No;I'm uncomfortable answering\}
\item
Have you ever participated in a study that involved deception?
\\ \{Yes; No; I'm uncomfortable answering\}
\item
Prior to participating in this study, had you heard about Facebook's `mood' study (the experiment that has the subject in many recent news stories).
\\ \{Yes; No\}
\end{packed_itemize}
%
We placed the question about prior knowledge of Facebook's experiment at the very end of our survey so as to avoid having this question taint responses to earlier questions.

\subsection{Payment}
We paid all respondents \$1 for the HIT regardless of their level of effort, answer quality, or time spent.  We also calculated a wage for each participant based on their time spent responding to the survey at an hourly wage of \$9.32 (the highest minimum wage of any state in the US), up to a maximum of \$3.11 for 20 minutes of time.  If the wage exceeded the \$1 paid for the HIT, we paid a bonus equal to the difference.  We paid bonuses after all surveys were complete---had we paid immediately and word spread, some respondents might have delayed completion.

\section{Experimental scenarios}
\label{sec:scenarios}

We created two scenarios for experiments from the past decade that were the subject of ethical debate in the research community, two scenarios for experiments that we had run and gauged participants' ethical response to at the time of the experiment, and one scenario (with ten variants) for the recent Facebook experiment.  In no description of these experimental scenarios did we mention that the experiment described was a real experiment or, in the case of the university studies, that it had been approved by an ethics board.

%These scenarios represent deception experiments run, in part, by members of our team.  Both received IRB approval from Carnegie Mellon University.  We selected these experiments because we have been collecting feedback from participants to gauge their feelings about whether these experiments should have been allowed to proceed.  Both experiments included consent forms, but both also included deceptions in which participants were not informed of the true purpose of the study until they had completed the experimental task.  In order to make these scenarios more closely resemble the controversial ones, we elided the consent forms entirely from our scenario descriptions.

\TreatmentSection{SP}
We wrote this experimental scenario around the ``Social Phishing'' experiment performed by researchers at Indiana University~\cite{Jagatic2007:SocialPhishing}.  In their experiment, researchers sent students phishing emails to see if they could be deceived into revealing their passwords on a website that impersonated a university system.  Some of the emails researchers sent were customized based on participants' public Facebook profiles. The researchers collected passwords from those who entered them and tested them against a university password database to determine if they were valid.  The exact wording of this scenario is in Appendix~\TreatmentNum{SP}.

We did not mention that participants were exposed to the experiment without their consent.

\TreatmentSection{BS}
The second experimental scenario describes an experiment to measure the economics of spam performed by researchers at the University of California~\cite{Kanich2008:Spamalytics}.  In this experiment, the researchers allowed a computer to be infected with software used to send spam.  The researchers then modified the spam to direct recipients to servers controlled by the researchers, instead of the spammers.  Thus, recipients of attackers' spam became unwitting participants in this study.  The exact wording of this scenario is in Appendix~\TreatmentNum{BS}.

As with the previous study, we did not explicitly state that spam recipients did not opt into the study via a consent form, though we did indicate that spam recipients who visited the impersonated store would not be informed that it was not the genuine store run by spammers.

\TreatmentSection{OSCS}
This scenario describes an experiment by researchers at Carnegie Mellon University and Microsoft Research to determine whether malicious websites can trick users into revealing their device (computer) password by mimicking (spoofing) security dialogs that are normally generated by the device's operating system~\cite{BravoLillo2012:MistakenIdentity}.  The researchers presented the experiment to participants as an evaluation of online gaming websites.  When participants visited a website run by the researchers, the researchers mimicked the operating system window used to download a software component.  The window indicated that it required the user's (participant's) device username and password to install the software component.  The researchers observed whether participants could be deceived to enter that information.  (Unlike the Indiana University phishing study, the researchers did not actually collect passwords without participants consent.)  The exact wording of this scenario is in Appendix~\TreatmentNum{OSCS}.

The experiment on which this scenario was run by a team that includes two authors of our ethical-response survey (and the paper you are reading now).  The experiment, which was led by Carnegie Mellon University and performed in collaboration with Microsoft Research, was was approved by the Institutional Review Board of Carnegie Mellon University.

Participants in the actual experiment had received a consent form explaining that they were part of a University experiment, though the consent form did not disclose that security was the focus of the experiment.  The researchers informed study participants of the deception during a debriefing at the end of the experiment.  We elided the presence of the consent form in order to make the scenario more similar to the other, more controversial, experiments described in this survey.

\TreatmentSection{WD}
This scenario describes an experiment by researchers at Carnegie Mellon University and Microsoft Research to improve security warning dialogs~\cite{BravoLillo2013:Attention}.  Like the previous study, it is a deception experiment in which researchers led participants to believe that online games were the focus of the study.  Unlike the previous study, users were not tricked into typing passwords.  Rather, they were shown a warning about the risk of installing software and the researchers tested to see whether participants could identify signs of danger in the warning.  Regardless of how participants responded to the install warning, no harm would come to them.  The exact wording presented of the scenario is in Appendix~\TreatmentNum{WD}.

As with the previous scenario, the experiment on which this scenario was run by a team that includes two authors of our ethical-response survey (and the paper you are reading now).  The experiment, which was led by Carnegie Mellon University and performed in collaboration with Microsoft Research, was was approved by the Institutional Review Board of Carnegie Mellon University.
Participants in the actual experiment had received a consent form explaining that they were part of a university experiment, though the consent form did not disclose that security was the focus of the experiment.  Further, the researchers collected data to monitor participants' ethical responses during the study to ensure harm was minimal.  We elided these facts in order to make the scenario more similar to the more controversial experiments described in this survey.

\TreatmentSection{F}
\label{scenario:Facebook}
\sscomment{Box this}
This scenario, presented in Figure~\ref{fig:facebookscenario}, describes Facebook's emotional contagion experiment, based on our understanding of the experiment from reading their paper.  The scenario focuses on facts about the experimental goals and methodology and so avoids touching on many issues that have been a subject of public debate.  Specifically, it does not discuss oversight, terms of service, or the participation of university researchers in the experiment.  As is consistent with the other scenarios, we do not explicitly state that the researchers did not obtain consent from participants.

However, many respondents did not receive this exact scenario (our control), but instead received one of the variants (treatments) that are described in the next section.

\begin{figure}[t]
\fcolorbox{boxbordercolor}{boxbgcolor}{
\begin{minipage}{0.95\columnwidth}
\footnotesize
Researchers at Facebook want to study whether users are more likely to share positive (happy) thoughts if their friends have been posting positive thoughts, and whether they are more likely to share negative (unhappy) thoughts if their friends have been sharing negative thoughts.
\begin{packed_itemize}
\item To increase the proportion of positive posts in some users' news feeds, the researchers will randomly exclude some fraction of friends' negative posts each time the news feed is loaded.
\item To increase the proportion of negative posts in some users' news feeds, the researchers will randomly exclude some fraction of friends' positive posts each time the news feed is loaded.
\item The researchers will use an automated algorithm to measure whether users' posts are of a positive or negative mood.
\item The researchers will publish the anonymized aggregate results of the experiment in a scientific paper.
\item Participants will not be identified and will remain anonymous.
\end{packed_itemize}
%
If the researchers are not allowed to perform this experiment, they will not be able to make a valid scientific determination of whether users' moods are affected by the moods of their friends' posts. Therefore, the researchers will not be able to produce features that might protect the moods of psychologically-vulnerable users.
\end{minipage}}
\caption{The experimental scenario description we used for Facebook's emotional contagion experiment.}
\label{fig:facebookscenario}
\end{figure}

\section{Treatments}

We created ten variants of the experimental scenario for the Facebook experiment.  We assigned respondents to scenario variants (treatments) at random with uniform probabilities assigned to each.

\TreatmentSection{C}
The control does not diverge from the facts of Facebook's experiment as we understood them, described in Section~\ref{sec:scenarios}.\ref{scenario:Facebook} and detailed in Figure~\ref{fig:facebookscenario}.

\TreatmentSection{WRNP}
We designed this scenario to test the hypothesis that respondents would be more disapproving and concerned by the removal of negative posts than by the removal of positive posts.  We thought respondents might see more harm in missing out on an opportunity to provide support to a friend in need, who had posted a negative post.

To construct this scenario we deleted the references to removing \textit{negative} posts for the purpose of increasing the proportion of \textit{positive} posts in the feed.  From the first paragraph, we removed the string: ``are more likely to share positive (happy) thoughts if their friends have been posting positive thoughts, and whether they''.  We also removed the first bullet point, which had stated: ``To increase the proportion of positive posts in some users' news feeds, the researchers will randomly exclude some fraction of friends' negative posts each time the news feed is loaded.''

\TreatmentSection{WRPP}
We designed this scenario to test the hypothesis that respondents might be particularly concerned with participants missing out on good news.  In this treatment, participants would only miss out on negative posts.

We deleted references to removing \textit{positive} posts for the purpose of increasing the proportion of \textit{negative} posts in the feed.  From the first paragraph, we removed the string: ``and whether they are more likely to share negative (unhappy) thoughts if their friends have been sharing negative thoughts''.  We also removed the second bullet point, which had stated: ``To increase the proportion of negative posts in some users' news feeds, the researchers will randomly exclude some fraction of friends' positive posts each time the news feed is loaded.''

\TreatmentSection{WP}
We created this scenario to test whether respondents would feel more or less favorably if the mention of a scientific publication were removed.  Specifically, we deleted the second-to-last bullet point of the scenario, which had stated ``The researchers will publish the anonymized aggregate results of the experiment in a scientific paper.''

\TreatmentSection{WIP}
We created this scenario to test whether respondents would feel less favorably about the experiment if there were no mention of potential for product improvement that might benefit users.
We removed the last sentence of the scenario, which had stated that a consequence of not allowing the research would be that ``the researchers will not be able to produce features that might protect the moods of psychologically-vulnerable users.''

\TreatmentSection{NA}
To test the hypothesis that respondents might respond more favorably to the experiment if the results would not be used for advertising, we created a scenario in which researchers promised this.  We appended one item to the list of bullet points.  It stated: ``The researchers promise in writing that the research findings will be used only to further science and improve the product for users.  The results will not be used to improve Facebook's advertising algorithms.''

\TreatmentSection{IP}
We hypothesized that respondents might be less concerned about researchers manipulating news feeds if the researchers had only added extra (bonus) posts, as opposed to removing that had been deemed relevant by Facebook's existing algorithms.

We changed the description of the study design so that, instead of hiding posts, the researchers would add negative or positive posts that otherwise would not have been deemed worthy of display on the news feed.
We rewrote the first two bullet points as follows:
\begin{itemize}
\item
To increase the proportion of positive posts in some users' news feeds, the researchers will randomly include additional positive posts that would otherwise have been deemed insufficiently relevant or unimportant.
\item
To increase the proportion of negative posts in some users' news feeds, the researchers will randomly include additional negative posts that would otherwise have been deemed insufficiently relevant or unimportant.
\end{itemize}

\TreatmentSection{IOP}
We hypothesized that respondents might be even less concerned if the added posts were only positive posts.
We started with the prior treatment (\TreatmentNum{IP}), and deleted from the first paragraph the string: ``and whether they are more likely to share negative (unhappy) thoughts if their friends have been sharing negative thoughts''.  We kept the first bullet point from the prior treatment (\TreatmentNum{IP}), which described increasing the proportion of positive posts, but deleted the second one, which had described increasing negative posts.

\TreatmentSection{UC}
To test whether respondents' opinions would change if the experiment were not identified as being conducted by Facebook, we replaced the third word of the scenario, ``Facebook'', with the phrase ``a social network''.

\TreatmentSection{T}
To test whether respondents might be have responded differently to the experimental scenario had it been conducted by Twitter, we replaced the third word of this scenario, ``Facebook'', with ``Twitter''.

