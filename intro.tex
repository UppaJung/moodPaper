\section{Introduction}
In evaluating the ethicality of an experiment, researchers and ethics boards must weigh the benefits of the study against potential risks---many of which are borne by participants.\sscomment{Aleecia would like a citation here.  Not sure what to cite.}  Alas, there is a great deal of guesswork in anticipating how participants and others will react to an experiment.  The information that researchers and ethics boards require in order to make sound judgements is hard to come by; researchers rarely share, or even measure, participants' feelings, concerns, and opinions about the ethicality of the experiments they take part in.  The rare instances in which we learn about the ethical consequences of experiments typically occur when concerns or harms are so serious as to come to the attention of the public.

In 2012, we began using surveys to identify disapproval and concern with experiments \emph{before} exposing participants to them.  We presented respondents\footnote{We refer to the group of Amazon Mechanical Turk workers who completed our survey as \emph{respondents}, as opposed to \emph{participants}, to prevent confusion between them and the participants in the experimental scenario our respondents were asked about.} with a series of short descriptions of experimental scenarios and asked questions to gauge their ethical response.  We wrote these short summaries with the goal of presenting the information salient to evaluating the ethicality of a study into a form that could be read and understood by a general audience within a minute.  Our first such survey caused us to re-evaluate how participants might react to a study we had planned (and received approval to conduct).  In light of our survey data, we concluded that the benefits to society of running the experiment no longer appeared to outweigh the risks.  We began publicly advocating for the prophylactic use of ethical-response surveys in 2013~\cite{BravoLillo2013:CREDS}.

In this new work, we ask what researchers at Facebook would have learned had they had the opportunity to prophylactically perform an ethical-response survey prior to commencing their 2012 emotional contagion experiment~\cite{Kramer2014:SocialContagion}.  In Facebook's experiment, researchers used an algorithm to remove posts from users' news feeds in order to determine whether a reduction in positive or negative posts from participants' friends would impact the emotional mood of posts made by participants themselves.  This experiment, which was published in June 2014, quickly became controversial---attracting criticism that the researchers and those overseeing their work presumably had not anticipated.

We performed an ethical-response survey on a convenience sample of \AllKnewAboutMoodStudyAll{A} workers on Amazon's Mechanical Turk who were based in the United States from July 2--4, 2014.  Of these, \AllKnewAboutMoodStudyno{A} reported not yet being aware of Facebook's emotional contagion experiment; these participants presented us with an opportunity to gauge opinions not yet influenced by media coverage and the evolving public reaction that has followed the publication of Facebook's experiment.

We presented participants with five experimental scenarios: one about the Facebook experiment and four about other experiments.  The details of the scenario related to the Facebook experiment varied between respondents whereas the other four did not.  We wrote our control scenario to describe the experiment based on our understanding of Facebook's experiment from reading their paper~\cite{Kramer2014:SocialContagion}.  We created nine other treatments in which we change facts about the scenario based on Facebook's experiment; we modified facts such as what manipulations Facebook's researchers had performed or even which company had performed the research.

Of the other four experimental scenarios, two described deception experiments that members of our team had led in the past and for which we had worked to measure participants' response to ethics questions asked after debriefing (though we elided the use of consent for these studies).  The final two scenarios summarized research from the past decade that had been the subject of ethical debate within the research community (without these elisions as the studies had been performed without consent).  All four experiments that these scenarios were based on had been conducted with approval from university ethics boards.

For each of the five abstracts we presented to each respondent, we asked two questions designed to gauge concern for participants and disapproval with allowing the study.  The participant \emph{concern} question asked whether the respondent would want someone they cared about to be included as a participant.  The \emph{disapproval} question asked whether the respondent believed the experiment should be allowed to proceed or not.

\preview{
Respondents who reported being previously aware of Facebook's experiment reported more concern for participants, and greater disapproval of proceeding with the experiment, than those who reported not being previously aware of it..  However, those who were not previously aware of Facebook's experiment expressed greater concern and disapproval when confronted with scenarios describing two controversial university experiments from the past decade than they did for Facebook's experiment.  We also identified variants of the Facebook-experiment scenario that reduced disapproval and concern, the most successful of which had the researchers manipulating news feeds by adding positive posts instead of removing (both positive and negative) posts.
}{}
%Had we provided different variants of the abstract, such as one in which the lack of consent was explicit rather than implicit, we might have seen different results.  Some variants appeared to reduce ethical concerns, but it seems doubtful they would have protected Facebook from criticism.  Some variants intended to address concerns may have even had the reverse effect.


%In the form of survey we previously proposed, and have used for our own studies in the past, researchers present respondents with short abstract descriptions of experiments.  For each abstract, the researchers ask questions to gauge respondents ethical disapproval and concern.  In our use of such surveys in the past, respondent feedback has caused us to determine that the

%Two issues of controversy in Facebook's study was the lack of oversight by an Institutional Review Board (IRB) and the lack of informed consent (beyond agreement to Facebook's terms of service).

%We investigate whether, short of obtaining consent or removing manipulations entirely, changes to experimental design might have been considered if a prophylactic survey had been used.

%From creds paper

%Researchers cannot always obtain the consent of those they observe.  In some studies it may be impractical to collect consent from everyone whose behavior might have impacted the data being collected.  Consider, for example, a study of network traffic flowing through a large ISP.  The aggregate traffic may be the result of millions of users' individual behavioral interactions.  Even if the researchers wanted to contact every individual involved for consent, the network traffic itself does not contain sufficient information to allow researchers to identify and contact them.

%In other studies, especially those involving observations of crime or victimization, those observed may consider being contacted for consent more harmful than simply allowing researchers to use the data.  For example, victims of password data breaches may approve of researchers' use of breach data to perform aggregate studies of password behavior, but might not approve of having their inbox cluttered with a request for consent from everyone who wanted to analyze the breach data--frequent and perhaps unnecessary reminders of their past victimization.

%In situations where consent from respondents cannot be obtained, ethics boards may require researchers to use \emph{surrogate} participants to determine whether participants would likely consent if asked, or to measure how participants might react to information.  Researchers may instruct surrogate participants to imagine themselves to be in the position of an actual participant and answer the question on that participant's behalf.  Surrogate participants can also provide advance feedback for studies in which consent may be obtainable, but where deception will render it uninformed and unacceptable levels of harm could occur before feedback can be obtained.

%There are high fixed costs to developing survey tools for reaching out to surrogate participants and collecting and analyzing data.  Surveying surrogate participants about multiple experimental designs at once significantly reduces the marginal cost for each additional experimental design.  Furthermore, when a single survey is used to investigate multiple experimental designs, researchers can compare the \emph{relative} ethical acceptability or repulsion of these designs.
