We surveyed \AllKnewAboutMoodStudyAll{A} workers on Amazon's Mechanical Turk to gauge their response to five scenarios describing scientific experiments---including one scenario based on Facebook's emotional contagion experiment.  Respondents$^{1}$ who reported being already aware of Facebook's experiment responded very differently to the scenario based on it than those who reported being unaware, so we focused on \AllKnewAboutMoodStudyno{A} respondents who reported being unaware.  We asked these respondents whether they would want someone they cared about to be included as a participant, interpreting an answer of `no' as indicating concern for participants.  A greater fraction of respondents were concerned about the two of the four scenarios inspired by university-approved experiments than expressed concern for Facebook's experiment.  We also asked whether the experiment should be allowed to proceed, interpreting a `no' answer as disapproval of the experiment.  A similar or greater fraction of respondents disapproved of the two more controversial scenarios based on university-approved studies as disapproved of the Facebook-experiment scenario.  We found a statistically significant reduction (for $\alpha=0.05$) in disapproval and concern for participants in a group of respondents shown a hypothetical variant of Facebook's experiment in which the manipulation performed by researchers was to insert extra positive posts into users' news feeds---instead of removing positive or negative posts based on treatment group. 