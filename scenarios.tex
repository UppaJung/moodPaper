\section*{Full text of scenarios}
\label{app:Verbatim}
\AppTreatmentSection{SP}
\fcolorbox{boxbordercolor}{boxbgcolor}{
\begin{minipage}{0.95\columnwidth}
\footnotesize
Phishing is an attack in which users are sent emails with a link to a fraudulent website in order to trick them into divulging their passwords. For example, some phishing emails appear to come from a user's bank and contain a link to a website that also appears to be the user's bank, but is actually controlled by the attacker. When the user types the password into the fake site, the attacker takes the password and can now login to the user's account.

University researchers want to quantify how much the success of a phishing attack would increase if the email its targets received appeared to come from someone the target user trusted—-a friend:
\begin{packed_itemize}
\item The researchers will send phishing emails to students with a link to a website that impersonates one of the university's websites.
\item The researchers will send half of the students an email that appears to be from one of the student's friends, who the researchers will identify by examining the student's Facebook profile.  The researchers will send the other half of students an email that appears to be sent by someone the student does not know.
\item If students enter passwords into the researchers' site, the researchers will, with the permission of the university, use the university's systems to verify that the passwords entered were valid passwords.
\item Afterwards, the researchers will notify students that this was a research study. They will inform offer students the opportunity to ask to have their data excluded from the study and to comment about the study on a blog.
\item The researchers will publish the anonymized aggregate results of the experiment in a scientific paper.
\item Participants will not be identified and will remain anonymous.
\end{packed_itemize}
If the researchers are not allowed to perform this experiment, they will not be able to measure how often users fall victim to phishing attacks.  Therefore, the researchers will not be able to publish recommendations to help users better learn to recognize such attacks.
\end{minipage}}

\AppTreatmentSection{BS}
\fcolorbox{boxbordercolor}{boxbgcolor}{
\begin{minipage}{0.95\columnwidth}
\footnotesize
Computer security researchers, seeking to understand the economic infrastructure that enables email spam, want to measure the rate at which spam emails result in purchases.

Conducting such research is challenging. Researchers would not want to send spam. Spammers are unlikely to divulge how successful their emails are in attracting purchases.
\begin{packed_itemize}
\item The researchers will allow one of their computers to become infected with software that is controlled by spammers, while the researchers maintain sufficient control of the computer to monitor how attackers are using it.
\item The researchers will alter the commands that the spammers send to the researchers' infected computer, replacing the link to the spammer's store with a link to a website run by the researchers that mimics the appearance of the spammer's store.
\item Without collecting payments or other personal information about those users who respond to the spam email seeking to make a purchase from the spammers, the researchers record the number of attempts made to purchase products from the store advertised by the spam.
\item The researchers will not inform users who receive the spam sent by attackers using the infected computer as this might cause users to behave differently or otherwise compromise the validity of the results.
\item The researchers will not inform users who visit the store to make a purchase that the store has been disabled or that their choice to make a purchase is being recorded.
\item The researchers will publish the anonymized aggregate results of the experiment in a scientific paper.
\item Participants will not be identified and will remain anonymous.
\end{packed_itemize}
If the researchers are not allowed to perform this experiment, they will not be able to empirically measure the effectiveness of spam emails and may not be able to produce or publish well-informed recommendations for technical or policy approaches to stopping spam.
\end{minipage}}

\pagebreak
\AppTreatmentSection{OSCS}
\fcolorbox{boxbordercolor}{boxbgcolor}{
\begin{minipage}{0.95\columnwidth}
\footnotesize
Computer security researchers want to learn the fraction of Internet users who fall for the tricks used by hackers to steal users’ passwords.

Conducting such research is challenging because if research participants know the attack is coming, or even that the study is about computer security, they may be less likely to fall for the tricks. The researchers thus plan to deceive participants as to the purpose of the human intelligence task (HIT) they will be asked to complete:
\begin{packed_itemize}
 \item During the task the researchers will replicate the techniques that hackers use to trick users into typing their passwords.
\item Unlike criminal hackers, the researchers will not actually steal, collect, or store the passwords that users type.
\item Afterwards, the researchers will present a detailed explanation of the deception to participants, reveal the true purpose of the study, and reassure participants that no passwords were actually stolen during the study.
\item The researchers will publish the anonymized aggregate results of the experiment in a scientific paper.
\item Participants will not be identified and will remain anonymous.
\end{packed_itemize}
If the researchers are not allowed to perform this experiment, they will not be able to measure how often users fall victim to attacks that target users' passwords. Therefore, the researchers will not be able to produce or publish recommendations that help users better learn to recognize such attacks.
\end{minipage}}

\AppTreatmentSection{WD}
\fcolorbox{boxbordercolor}{boxbgcolor}{
\begin{minipage}{0.95\columnwidth}
\footnotesize
Computer security researchers want to measure different techniques for presenting security warnings.

One challenge in studying security decision making is that if participants are made aware that researchers are studying their security behavior, or become aware of it, they are likely to behave differently than they normally would. The researchers thus plan to deceive participants as to the purpose of the human intelligence task (HIT) they will be asked to complete:
\begin{packed_itemize}
 \item The researchers will give participants a task unrelated to security, but that will cause participants to encounter a security warning.
\item While the warning will create the illusion that the participant is facing a security risk, the researchers will not actually expose participants to any real security risks.
\item The researchers will measure how different ways of presenting a warning may make that warning more or less effective in convincing users to avoid a risk.
\item At the conclusion of the experiment, the researchers will present a detailed explanation of the deception to participants, reveal the true purpose of the study, and reassure participants that they were never at any real risk.
\item The researchers will publish the anonymized aggregate results of the experiment in a scientific paper.
\item Participants will not be identified and will remain anonymous.
\end{packed_itemize}
If the researchers are not allowed to perform this experiment, they will not be able to measure the effectiveness of different designs for computer security warnings. Therefore, the researchers will not be able to produce or publish recommendations to improve the effectiveness of future security warnings.
\end{minipage}}
\normalsize
