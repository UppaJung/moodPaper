\section{Treatments}

We created ten variants of the experimental scenario for the Facebook experiment.  We assigned respondents to scenario variants (treatments) at random with uniform probabilities assigned to each.

\TreatmentSection{C}
The control does not diverge from the facts of Facebook's experiment as we understood them, described in Section~\ref{sec:scenarios}.\ref{scenario:Facebook} and detailed in Figure~\ref{fig:facebookscenario}.

\TreatmentSection{WRNP}
We designed this scenario to test the hypothesis that respondents would be more disapproving and concerned by the removal of negative posts than by the removal of positive posts.  We thought respondents might see more harm in missing out on an opportunity to provide support to a friend in need, who had posted a negative post.

To construct this scenario we deleted the references to removing \textit{negative} posts for the purpose of increasing the proportion of \textit{positive} posts in the feed.  From the first paragraph, we removed the string: ``are more likely to share positive (happy) thoughts if their friends have been posting positive thoughts, and whether they''.  We also removed the first bullet point, which had stated: ``To increase the proportion of positive posts in some users' news feeds, the researchers will randomly exclude some fraction of friends' negative posts each time the news feed is loaded.''

\TreatmentSection{WRPP}
We designed this scenario to test the hypothesis that respondents might be particularly concerned with participants missing out on good news.  In this treatment, participants would only miss out on negative posts.

We deleted references to removing \textit{positive} posts for the purpose of increasing the proportion of \textit{negative} posts in the feed.  From the first paragraph, we removed the string: ``and whether they are more likely to share negative (unhappy) thoughts if their friends have been sharing negative thoughts''.  We also removed the second bullet point, which had stated: ``To increase the proportion of negative posts in some users' news feeds, the researchers will randomly exclude some fraction of friends' positive posts each time the news feed is loaded.''

\TreatmentSection{WP}
We created this scenario to test whether respondents would feel more or less favorably if the mention of a scientific publication were removed.  Specifically, we deleted the second-to-last bullet point of the scenario, which had stated ``The researchers will publish the anonymized aggregate results of the experiment in a scientific paper.''

\TreatmentSection{WIP}
We created this scenario to test whether respondents would feel less favorably about the experiment if there were no mention of potential for product improvement that might benefit users.
We removed the last sentence of the scenario, which had stated that a consequence of not allowing the research would be that ``the researchers will not be able to produce features that might protect the moods of psychologically-vulnerable users.''

\TreatmentSection{NA}
To test the hypothesis that respondents might respond more favorably to the experiment if the results would not be used for advertising, we created a scenario in which researchers promised this.  We appended one item to the list of bullet points.  It stated: ``The researchers promise in writing that the research findings will be used only to further science and improve the product for users.  The results will not be used to improve Facebook's advertising algorithms.''

\TreatmentSection{IP}
We hypothesized that respondents might be less concerned about researchers manipulating news feeds if the researchers had only added extra (bonus) posts, as opposed to removing that had been deemed relevant by Facebook's existing algorithms.

We changed the description of the study design so that, instead of hiding posts, the researchers would add negative or positive posts that otherwise would not have been deemed worthy of display on the news feed.
We rewrote the first two bullet points as follows:
\begin{itemize}
\item
To increase the proportion of positive posts in some users' news feeds, the researchers will randomly include additional positive posts that would otherwise have been deemed insufficiently relevant or unimportant.
\item
To increase the proportion of negative posts in some users' news feeds, the researchers will randomly include additional negative posts that would otherwise have been deemed insufficiently relevant or unimportant.
\end{itemize}

\TreatmentSection{IOP}
We hypothesized that respondents might be even less concerned if the added posts were only positive posts.
We started with the prior treatment (\TreatmentNum{IP}), and deleted from the first paragraph the string: ``and whether they are more likely to share negative (unhappy) thoughts if their friends have been sharing negative thoughts''.  We kept the first bullet point from the prior treatment (\TreatmentNum{IP}), which described increasing the proportion of positive posts, but deleted the second one, which had described increasing negative posts.

\TreatmentSection{UC}
To test whether respondents' opinions would change if the experiment were not identified as being conducted by Facebook, we replaced the third word of the scenario, ``Facebook'', with the phrase ``a social network''.

\TreatmentSection{T}
To test whether respondents might be have responded differently to the experimental scenario had it been conducted by Twitter, we replaced the third word of this scenario, ``Facebook'', with ``Twitter''.
